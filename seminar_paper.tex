\documentclass[conference]{IEEEtran}
\IEEEoverridecommandlockouts
% The preceding line is only needed to identify funding in the first footnote. If that is unneeded, please comment it out.
\usepackage{cite}
\usepackage{amsmath,amssymb,amsfonts}
\usepackage{algorithmic}
\usepackage{graphicx}
\usepackage{textcomp}
\usepackage{xcolor}
\usepackage{bm}
\def\BibTeX{{\rm B\kern-.05em{\sc i\kern-.025em b}\kern-.08em
    T\kern-.1667em\lower.7ex\hbox{E}\kern-.125emX}}
\begin{document}

\title{Crowd Counting Using WiFi\\
%{\footnotesize \textsuperscript{*}Note: Sub-titles are not captured in Xplore and
%should not be used}
}

\author{\IEEEauthorblockN{1\textsuperscript{st} Andreas Rayzik}
\IEEEauthorblockA{\textit{dept. name of organization (of Aff.)} \\
\textit{name of organization (of Aff.)}\\
City, Country \\
email address}

}

\maketitle

\begin{abstract}
By using WiFi as a medium for estimating the number of people in a room or in a certain area, a great potential of cost-saving and simplification of deployment can be leveraged.
\vspace{80mm}
\end{abstract}

\begin{IEEEkeywords}
WiFi, IoT, people counting, crowd counting
\end{IEEEkeywords}

%Length: 0,75 – 1 page
\section{Introduction}
The main motivations behind performing crowd counting in buildings are the applications of energy efficiency, customer guidance in retail stores and traffic optimization "smart" cities. For example, traffic density esimation, travel time measurement or gap counts between successive drivers are possible. By estimating the number of people in a building, the demand for energy-intensive appliances like air conditioning or heating can be calculated and regulated \cite{Agarwal}. By reducing the operation of the appliances for a low number of people, a substantial amount of energy can be saved compared to an unregulated operation with a fixed schedule. In the cited work in particular, potential energy savings from 10\% to 15\% are reported.
\par
In customer guidance applications, people could receive information or advertisement based on their proximity to specific products in a store. With the user's consent, this data could be connected to personal characteristics to increase the relevancy of the provided information. For the retail operator, the motivation is to improve user experience and to increase sales counts. For the customer, the additional information can lead to better informed purchase decisions and a feeling of being guided while navigating a store.
In \cite{RetailStores}, the conclusion is that with increased traffic, the store sales volume does not scale linearly. Another interesting finding is the intra-day and inter-day traffic variability correlate with a diminishing store sales volume.
\par
Motivations for using WiFi in particular are the low cost of using off-the-shelf components like consumer WLAN routers and transceiver sticks. Off-the-shelf hardware comes with approval by the Federal Communications Commission, which allocates frequency ranges and determines maximum signal strengths to protect users from harmful radiation. In the official document for the FCC's frequency allocations \cite{fcctable}, it can be seen that the 2.4GHz band used among others for "Wireless Communications" and "ISM Equipment". ISM stands for Industrial, Scientific and Medical equipment. It is meant to be distinct from the purpose of telecommunications. Examples for devices are microwave ovens or medical diathermy machines.
If one were to develop one's own wireless technology for the problem, an assignment of the used frequency range and maximum signal strength would first be mandatory to apply the solution in public places. The process of assignment and checking is complex and time-consuming. This is the reason why using off-the-shelf hardware can save a large amount of resources.
\par
The following work will provide a history of approaches described in research literature. Additionally, a quantitative comparison between the approaches will be shown. Further, the work tries to give an explanation why the problem has not been solved yet despite many years of research. The main contribution of this paper will be a comparison to approaching the crowd counting problem with either WiFi technology, camera sensors or RFID sensors. This comparison will take the factors of accuracy, cost, privacy and perceived privacy by the target person into account. The work is concluded by a Discussion section, that points out assumptions and simplifications being made in other works. Why and how these might make a transition to a real-world scenario infeasible will be laid out. In addition to this, there will be a general comparison to device-based methods. All of the methods explained before are device-free, which means that the people to be counted are not required to carry any sort of devices with them.


\section{Challenges}
The hypothetical best solution would be a system with a deployment as easy as setting up one low-cost device per building, in an arbitrary location inside the building, while requiring no manual calibration and minimal automatic calibration. Research has shown that reaching these goals comes with a complicated trade-off and requires creativity and high engineering ingenuity. 
\par
In any environments that are not free of obstructions like walls, furniture, people or animals, WiFi signals propagate in a manner that is very hard to predict. Areas of constructive inferences and destructive inferences can change the signal strength for a receiver significantly. (TODO: How much? 0 to double?) When setting up a WLAN router for a home or a public building, positioning of the router is important. Slight changes in position can mean large changes in signal coverage for the building. Especially coverage of frequented areas is desirable. As WiFi signals consist of electromagnetic radiation, its interference pattern is shaped by its wavelength. The wavelength $\lambda$ can be calculated by: $$\lambda = \frac{c}{f}$$ With the speed of light $c$, the wavelength evaluates to approximately 12.5cm for 2,4GHz WiFi and approximately 6.25cm for 5GHz WiFi respectively.
\par
Designing a crowd counting implementation that scales to large events with thousands or even millions of people adds another large challenge. The approach would need to scale not only to the number of people, but also to the density of crowds and to the size of the area. For an event that persists over a number of days, it needs to be robust and low in maintenance. Constrained by the limited range of WiFi signals, scaling to a larger area for a big event makes using multiple transceivers unavoidable. The transceivers need not be disturbed by each other's interference. At the same time, they are required to be in close enough proximity to each other in order to cover the whole area and leave uncovered paths that might miss subjects to be counted.

\par
Table \ref{table_overview} provides an overview of the features of the described techniques for crowd counting that have been developed, sorted chronologically.
\begin{table*}
\caption{Overview of techniques}
\label{table_overview}
\centering
\begin{tabular}{c || c || c || c || c}
\hline
\bfseries \textbf{Author} & \textbf{Publication} & \bfseries\textbf{\# Nodes} & \bfseries \textbf{Max. \# People} & \bfseries \textbf{Calibration} \\
\hline
Yuan et al. & 2011 & 16 & Low/Medium/High & Not mentioned \\
\hline
Xu et al. & 2013 & 22 & 4 (+ location) & Little (Empty room RSS meas.) \\
\hline
Xi et al. & 2014 & 38 & 30 & Extensive \\
\hline
Yoshida/Taniguchi & 2015 & 11 & 7 & Extensive \\
\hline
Depatla et al. & 2015 & 2 & 10 & Little \\
\hline
Domenico et al. & 2016 & 2 & 7 & Extensive \\
\hline
Depatla/Mostofi & 2018 & 2 & 20 & Very little
\end{tabular}
\end{table*}
The feature "calibration" is meant to encompass methods like measuring the signal strength for specific densities of people beforehand, 
As can be seen in \ref{table_overview}, a rough trend has been to reduce the number of nodes needed as well as the calibration necessary.
\par
Bla
\subsection{Crowd Density Estimation Using Wireless Sensor Networks}
Yuan et al. \cite{Yuan} use Wireless Sensor Networks (WSN) to achieve an estimate of the crowd density in a room. A calibration step and a detection step are being iterated in pairs. They validate their approach by deploying a network of 16 wireless sensors. 
\par
The network uses the Collection Tree Protocol (CTP) \cite{ctp}. The CTP is required to collect the data gathered by the sensors reliably and efficiently. Moreover, it aims to minimize energy consumption to extend the sensors' battery lives. It extends the distance vector routing protocol by three additional mechanisms in order to adapt to the special conditions in highly dynamic wireless networks.
\begin{itemize}
\item Information from the physical layer, network layer and link layer are combined to estimate the link quality between two nodes with a resolution of four bits.
\item To detect datapaths that have become unreliable, data packets contain the transmitter's local cost estimate. The expected transmissions (ETX) can be used as a cost metric. When a packet is being forwarded, the transmitter's cost is expected to always be greater than the forwarding node's own cost. A deviation from this assumption hints at a loop in the network.
\item Adaptive Beaconing deals with the compromise of choosing a large or small interval for sending control packets. A large interval saves bandwidth and energy, but makes the network slower to react to adverse wireless dynamics. Adaptive beaconing changes the interval according to the consistency and problem rate in the network.
\end{itemize}
In the detection step, the K-means algorithm is used to cluster the data from the WSN into density levels of low, medium of high. The thresholds for these levels are set arbitrarily and can be changed. For the number of clusters K, the best convergence was achieved with $K=2$.


\subsection{SCPL: Indoor Device-Free Multi-Subject Counting and Localization Using Radio Signal Strength}
Xu et al. \cite{Xu} use the long-range property of radio signals to deploy a counting technique in two large indoor settings with 150$m^2$ and 400$m^2$. They can detect up to four subjects which is a low number compared to the other works, but it includes the subject's location. The abbreviation SCPL refers to the way of counting the number of subjects sequentially while doing the localization in parallel. They model the human trajectories as a state transition process and use Conditional Random Fields (CRF) to attack the localization problem.
\par
As a calibration, the RSS values for every link are measured when the rooms are empty. This data can then be used to determine how much a subject's presence alters the RSS value in a cell. For the presence of multiple subjects, a hypothesis is being made that it will "affect a larger number of spatially distributed radio links, [...] also lead to a higher level of RSS change on these links." Instead of using the RSS mean difference as a metric, the authors propose the absolute RSS mean difference. As the reason for this, they state that the additional presence of another subject might not always weaken a link. Due to multipath effects, it might actually strengthen a link. Thus, the RSS mean difference is not always positive. The absolute RSS mean difference represents the total energy change more suitably.
\par
The researchers exploit the fact that human movement constitutes a continuous trajectory. They model the movement as a CRF, which is a type of discriminative undirected probabilistic graphical model \cite{crf}. CRFs are comparable to hidden Markov models (HMMs), but possess for example the advantage to be able to relax strong independence assumptions. CRFs models the probability $$p(\bm{Y}|\bm{X})$$ for the jointly distributed random variables $\bm{Y}$ and $\bm{X}$. The aim is to maximize the likelihood that a subject is located in the same cell $i$ that was estimated. The cells correspond to the states there were defined as $\bm{Y}$.

\subsection{Electronic Frog Eye: Counting Crowd Using WiFi}
In a work from 2014 by Xi et al. \cite{Xi}, Channel State Information (CSI) is used to outperform state-of-art approaches at that time. As CSI is very sensitive to environment radiation, they figuratively compare it to the eye of a frog, which is also sensitive in the sense that it needs to react quickly to incoming prey. CSI is distinguished from Received Signal Strength (RSS) as it provides another level of information. It draws its information more closely from the physical layer. It adds values for the attenuation and phase shift for every subcarrier. These values help to describe the scattering, fading and power decay with distance properties of the wireless communication. As a metric, they formulate the Percentage of nonzero Elements (PEM). It relates to the dilated CSI matrix. The basic model for a narrowband flat-fading channel in the frequency domain is
$$ Y = HX + N $$
with $Y$ and $X$ denoting the receive and transmit vectors, $H$ as the channel matrix CSI and a noise vector $N$. Subsequently, the value of H results in 
$$ \widehat{H} = \frac{Y}{X} $$
\par
The authors continue to explain the compromise that needs to be made in order to measure the variation in CSI. The variation correlates with the number of people to be counted. If a large size sample space is being employed, the long sampling time would degrade system performance. If a small size sample space is selected, the variance becomes statistically insignificant. As a solution, the metric PEM is chosen to represent the current CSI variance. The CSI amplitude values are transformed into a two-dimensional matrix, the matrix is dilated, and then the number of non-zero elements in that matrix are counted.
\par
The paper goes on to discuss the scalability of the described approach. To cover a larger area for the crowd counting, more devices need to be added to the system to overcome their limited range. This leads to mutual interferences. As the utilised 802.11n standard offers 20MHz of bandwith or 40MHz with channel bonding, this leaves some room to cope with interferences. Nevertheless, a truncation threshold $\alpha$ is set that aims to mitigate the mutual interference between two neigboring grids. It is shown that setting $\alpha = 0.3$ reduces the effective sensing area between transmitter and receiver to an area that only revolves closely around the line of sight.
\par
In the experimental validation, they achieve good accuracy. The estimation error is less than 2 persons for an indoor environment and about 2 persons of error in an outdoor environment. In a test where the scalability to a larger area was evaluated, the accuracy is at less than 2 persons using a modified method.

\subsection{Estimating the number of people using existing WiFi access point in indoor environment}
Yoshida and Taniguchi \cite{Yoshida} compare a support vector regression-based method and a linear regression-based method in their paper. Regarding the majority of other techniques, this stands out as more of a Machine Learning approach than a Signal Processing approach. They consider the use of existing WiFi devices such as computers, televisions or printers for estimating the number of people. Likewise, they assume that they can access the devices' RSSI readouts by installing a new application to the device or performing a custom firmware update.
\par
The linear regression-based method is defined as
\begin{equation}
	\hat{m}_{k} = a_0 + \sum\limits_{n_i \in \mathcal{N}} a_i r^{\textrm{mid}}_{k,i}
\end{equation}


For the support vector regression method, the radial basis function kernel with $\sigma = 0.08$ is being adopted.
\par
The evaluation is divided into three situations. The first one is deemed presence/absence of people and only distinguishes between these binary states. The second one estimates a degree of congestion which was defined as three categories. Only the third estimation deals with the absolute number of people. These situations ascend in order of difficulty. As expected, the accuracy rates decrease in that same order. For every situation, the accuracy of the SVR-based method is consistently and significantly higher than for the LR-based method.

\subsection{X-Ray Vision with Only WiFi Power Measurements Using Rytov Wave Models}
Depatla, Buckland and Mostofi published a paper in 2015 \cite{DepatlaMostofi2015} where they introduce modelling of wave propagation to enable high-resolution see-through imaging. They apply their technique to unmanned, autonomously moving robots and discuss how they improved on the robot positioning and antenna alignment. As this is out of the scope of the current paper, it shall not be explained in further detail. For the wave propagation modelling, Rytov approximations are considered suitable. They are a linearising approximation to Maxwell's equations. Scattering effects are included. They explain that the Wentzel Kramer Brillouin (WKB) approximation often used in X-ray is not suitable for their purposes. It makes an assumption that only holds at high frequencies. The 2,4GHz microwave frequency is low compared to X-ray frequencies. In particular, they state that "Throughout[sic] this paper, high frequency refers to the frequencies at which the size of inhomogeneity of objects is much larger than the wavelength." They go on to lay out the Rytov approximation. It is followed by an explanation of an intensity-only variant of the Rytov Approximation. This is narrowed down further to an intensity-only LOS approximation, that results in the final equation $$P_{LOS} = A\Gamma$$
$P$ is the received power in dBm at $r$. $A$ is a two-dimensional matrix with entries 1 and 0 for cells that are along the LOS for a measurement $i$ or not along the line, respectively. $\Gamma$ is a vector of complex numbers $\alpha$, where $\alpha$ denotes the slowness of the medium that is penetrated by the radio wave at a position vector $r$.
\par
Because there is a much larger number of unknowns in the linearised equations than the number of wireless measurements can provide for, the system is substantially underdetermined. To alleviate this problem, sparse signal processing is introduced. The fact that the spatial variation of objects to be imaged is sparse is exploited. The authors utilize the MATLAB-based solver TVAL3 (TV Minimization by Augmented Lagrangian and Alternating Direction Algorithm). TV stands for Total Variation. It is a mathematical concept that describes roughly an infinitesimal version of the absolute value.
Figure ~\ref{figRytovLOS} is taken from the paper and shows both the experimental setup in the top-left, as well as the performance of the Rytov approximation over simple LOS approximation.
\begin{figure}[htbp]
\centerline{\includegraphics[scale=0.4]{figRytovVsLOS.PNG}}
\caption{Comparison of the Rytov approximation over LOS approximation in the experimental setup \cite{DepatlaMostofi2015}}
\label{figRytovLOS}
\end{figure}
The images visualise very clearly how the Rytov approximations improve the measurement over LOS, which would be hard to describe in words otherwise.
\par
The focus in the paper lies on capturing high-resolution images through walls, using WiFi. To adapt this approach to crowd counting, the sampling rate of the measurements would need to be increased to be able to capture moving persons. Currently, the approach is computationally intensive and takes 3.6 seconds for the Rytov approach on a 3.7GHz CPU. This corresponds to a sampling rate of 0.28Hz. A post-processing step would also have to be added to distinguish people from material obstacles. Thirdly, the number of people to be counted relative to the room size would need to be small enough, since the Rytov approximation exploits the sparsity of spatial variation.

\subsection{Crowd Counting through Walls}
In 2018, Depatla and Mostofi \cite{DepatlaMostofi2018} refined their previous approach to be able to increase the maximum number of people to be counted from 10 to 20. At the same time, the calibration needed was reduced further. They extend their concept to be applicable through walls instead of inside walls. However, their new technique comes with a number of assumptions that make it difficult to be translated to a real-world scenario. A general concept of their approach is depicted in \ref{concept}. The transceiver Tx and the receiver Rx are placed on opposite sides outside the walls of the target room. The direct path between them forms the line of sight (LOS). Up to 20 persons move inside the room. A person's movement is described by the movement speed in x and y coordinates and the angle of direction $\theta$. When one or multiple persons cross the line of sight, the signal is being blocked and attenuated. Also, the presence of persons in the room inevitably leads to multipath effects.
\begin{figure}
\label{concept}
\end{figure}
The researchers define the LOS crossing as an event $E$ and the inter-event times as $T_n$. They then define the discretised motion of a person as a Renewal-type process. Renewal theory is a generalization of Poisson processes, where the holding times can be arbitrary. In our case, the inter-event times $T_n$ represent the holding times. The theory of Poisson processes can be traced back to Markov processes. The inter-event times are identically distributed, but not independent. The authors prove that the probability mass function (PMF) of the inter-event times is an implicit function of the number of people $N$.
\par
They go on to experimentally validate their approach by a setup using a D-link WBR-1310 router as WiFi transceiver, a TP-Link USB card as WiFi receiver and a Raspberry Pi board for controlling the measurements and storing the obtained RSSI values. All of the devices are cheap, off-the-shelf products. Since the measurement run with a sampling rate of 20Hz over the course of 300 seconds, the Raspberry Pi needs to store 6000 values. The experiment is conducted for different numbers of people $$N = (1, 3, 5, 7, 9)$$ up to $N=20$. Further, different kinds of rooms are being tested with varying dimensions and wall materials. Test subjects were also instructed to follow different walking speeds. The authors conclude that their approach achieves and accuracy of less than 1 person difference 81\% of the time and less than a 2 person difference 100\% of the time.
\par
Compared to their previous work from 2015, the new technique is very different. It uses no modelling of wave propagation and thus achieves a much higher sample rate. It uses stationary WiFi transceivers instead of transceivers mounted on autonomous robots. It deals directly with the task of counting people instead of capturing images with radio waves. Both works have in common the fact that they send radio waves through walls.
\section{Discussion}
\subsection{Ability to alter wireless devices' firmware}
The paper by Yoshida and Taniguchi \cite{Yoshida} offers an interesting perspective to the challenge of WiFi crowd counting by pointing out that there are already a high number of WiFi-capable devices in our surroundings. Computers, televisions and printers are examples for these devices. With the industry trend of IoT, there will conceivably be even more devices in the future and whole new categories of devices that connect over WiFi. However, the authors assume that the firmware of these devices is customisable or the software on top of it provides enough access to get RSSI readouts. This assumption is questionable, because often times the device manufacturers generally do not want to give their customers the ability to freely alter their device's firmware. They prefer do protect to what their perceive as their intellectual property, and have the opportunity to sell new firmwares with added features in new product lines. Letting the consumers flash new firmwares themselves might enhance an old device's capability and prevent the consumer from buying a new one, thus negatively impacting sales. OpenWrt \cite{OpenWrt} is a community-driven project that strives to restore the customer's ability to freely adapt their wireless device's firmware to their needs. The official website lists 24 developers and 3 people working on the Wiki Documentation currently involved in the project. Considering that OpenWrt is a collection of volunteers with no revenue or central funding, it can be called a large and relevant project. Nevertheless, the website's list of compatible devices is limited to the category of WLAN routers and the list, while extensive, shows only subset of all devices available. This shows that the ability of changing and flashing a firmware for a device must not be taken for granted.

\subsection{Assumptions about people's movement and transition to real-world scenario in the 2018 paper by Depatla and Mostofi}
All of the works presented are limited by the number of people that can be counted at maximum, by the size of the room or other conditions. This makes it difficult to apply them to real-world scenarios. The work by Depatla and Mostofi ~\cite{DepatlaMostofi2018} makes particularly strong assumptions. These assumptions and to what degree they prevent the transition to a real-world scenario will be discussed in the following.
\begin{itemize}
\item Random movement patterns by the subjects are assumed. This might be applicable to a situation in which a subject is searching for a certain object, with few clues to where said object might be located. It could also hold in a place like a cafeteria, where people are moving around to greet several acquaintances and search for a free seat. The same goes for a hallway, which connects many places that represent potential starting points or end points for movement. It is problematic in a case in which people partially or in their entirety move as a group with a common goal. Since the authors use a uniform distribution for their motion model, the problem could potentially be tackled by switching to a non-uniform distribution which is adapted to the room and common goals in the room and its context.
\item Subjects are assumed to move independently from each other. This is closely related to the first assumption. It ignores the possibility that a person's movement might influence the movement of others and lead to group dynamics or swarm-like behaviours. This is especially meaningful in a setting where there is a hierarchy among the subjects with one or multiple leaders guiding a group.
\item The previous points lead to the problem of the line of sight (LOS) not actually being crossed. The event of LOS crossing is essential to the authors' technique of estimating the number of people. However, when instead of random, independent movement there is a fixed goal or social group dynamic, the LOS might not be crossed at all. This weakness could be mitigated by adding more pairs of transmitters and receivers which each introduce another LOS per pair.
\item The measurement is required to take at least 300 seconds. During this time, continuous random and independent movement is assumed and no subjects may leave or enter the room. Scenarios with a changing amount of people during the measurement are explicitly stated by the authors as a goal for future work.
\item Another limitation is the number 20 as the maximum of people to be counted. This number is higher than most of the other values in table \ref{table_overview}. Still, when considering a real-world scenario, it is a significant limit and stands in contrast to the advantages of requiring only two wireless nodes and very little calibration. The authors have not provided experimental results for more than 20 people. It is reasonable to expect their technique to work for a higher number when a lower accuracy is acceptable. 
\end{itemize}
Despite all of the problems mentioned, the work makes arguments to how an application in a real-world setting might function. The researchers show that their counting accuracy is robust to different walking speeds. Speeds ranging from casual walking to running have been tested. Moreover, diverse types of rooms have been tested with different dimensions and different wall materials like concrete, plaster or wood.

\subsection{Considerations about sampling rates}
In general, to facilitate crowd counting for moving subjects, are reasonably high sampling rate is required. Else, the movement of people during one measurement sample could introduce too much noise and make it inaccurate. Subjects that enter and leave an area quickly might not be counted at all. This makes approaches using radio wave modelling difficult to implement. In a three-dimensional setting, models for radio waves are computationally intensive. When high resolution is required, the computational costs are further enlarged. To make such a technique feasible, a highly optimised and parallelised algorithm running on a GPU or FPGA is probably needed. GPUs are processors that offer a much higher degree of potential parallelism compared to a CPU. The disadvantage is that designing a parallelisable algorithm is challenging and programming it for a GPU is difficult as well. FPGAs are chips in which the circuitry itself is programmable for the user. The offer a higher degree of flexibility in contrast to GPUs, but therefore inevitably also increase the complexity of software design.

\subsection{Importance of the amount of calibration required}
The amount of calibration needed for the different techniques compared in table ~\ref{table_overview} does not finally decide over the usefulness of a given technique. Rather, it needs to be put into context with the type of application and area. For approaches that scale to areas and number of people for large events, a high calibration amount is most likely appropriate and worth its effort. On the other hand, for applications where the focus lies on quick deployment and easy setup, an extensive calibration interferes with the usefulness of the approach. Some of the techniques need calibration in an empty room, like the one by Xu et al.\cite{Xu}. If the room is empty at the time of deployment, this calibration step does not add further complications. Calibration methods that require a known number of subjects to be moving randomly or in defined patterns across the area, it is conceivable to automate this step using autonomous robots. The robots will need to be similar to humans in their dimensions and movement speed.

\subsection{Scalability considerations}
Scaling an approach to a high number of people as it might be prevalent in a real-world scenario adds an additional challenge. Of all the methods presented, the one by Xi et al. \cite{Xi} reports the highest maximum number of people countable, which is 30. However, one of their result diagrams shown in figure \ref{figFrogeyeConvergence} very clearly that the approach will not scale beyond this number. In the diagram, the percentage of non-zero elements metric (PEM) that they introduce and the estimation error are plotted along the axis of ordinates. Along the axis of abscissae, the number of people from 0 to 30 is presented. It is obvious that the PEM converges at about 20 people. At the same time, the estimation error rises in a seemingly exponential manner.
\begin{figure}[htbp]
\centerline{\includegraphics[scale=0.8]{figFrogeyeConvergence.PNG}}
\caption{The maximum distinguished number of people \cite{Xi}}
\label{figFrogeyeConvergence}
\end{figure}

Their CSI measurements for increasing numbers of people illustrate the reason for this behaviour \ref{figFrogeyeCsiVariation}. The diagrams show CSI measurements for 0, 2, 4 and 6 moving persons. While in the case of 2 and 4 persons the CSI variation is still perceivable by looking at plot, the case of 6 persons makes it apparent that the pattern starts to resemble a noise pattern.
\begin{figure}[htbp]
\centerline{\includegraphics[scale=0.6]{figFrogeyeCsiVariation.png}}
\caption{CSI measurements of one subcarrier with different moving people \cite{Xi}}
\label{figFrogeyeCsiVariation}
\end{figure}

\section{Comparison of device-free and device-based methods}
All of the methods discussed so far have been device-free. This means that the subjects to count are not required to carry any sort of device with them. The method relies solely on the property of human bodies to reflect and attenuate WiFi signals, thereby altering the signal strength in the receiver. In contrast, device-based methods work by either assuming that the subjects are carrying a device (usually a smartphone) or handing out devices to them to carry prior to the counting phase.
\par
In the case using smartphones to identify people, this implies complications for user privacy. Smartphones are personal devices, usually used by one person exclusively, containing sensible and potentially valuable personal data. The common approach is to have WiFi or Bluetooth beacons which can sense a smartphone. Smartphones typically scan for open WiFi hotspots and Bluetooth devices at regular intervals. These intervals vary from device to device and by software configuration. If the smartphone is currently in some battery saving state, the interval could be lengthened to the point where a user could walk by a beacon without his smartphone sending a signal to it. In general though, the beacon will catch the smartphones in its surroundings. To protect the users' privacy, the beacon will only save the device's MAC address, which is the address on the physical layer. The MAC address is distinct from a phone's IMEI (International Mobile Equipment Identity). It is not linked to the phone call number in any way. To anonymise the users further, the MAC addresses should be hashed by a modern hashing algorithm like SHA-256. A hashing algorithm is an algorithm that takes an input of arbitrary length and outputs a code of fixed length. It is designed to be infeasible to reverse, so the output code cannot be converted back to the original input. However, the limited number of possible MAC addresses poses a problem. There are $2^48$ possible addresses. With modern, parallel cloud computing and a large enough budget, it is possible to compute the hashes for every possible address in advance and store them in a so-called rainbow-table. The table could then be used as a quick lookup and defy the purpose of the hashing function. During the recent Cryptocurrency mining trend, special ASIC chips have been designed that are especially fast and efficient at computing hash functions \cite{crypto}. If an attacker has access to these kind of devices, he might be able to compute a rainbow table with a substantially reduced budget. A way of mitigating this threat is to add random code to the MAC addresses before hashing. In cryptography, this is called Salting. Nevertheless, the salts need to be stored and can also be stolen by an attacker.

\section{Elaboration of privacy and perceived privacy by the user}
Device-based methods protect the users' privacy, if only the MAC addresses are collected and hashed with a modern hashing algorithm. However, the users might perceive their privacy as being intruded when it is explained to them that their smartphone is being detected by a beacon. Users are typically not conscious about the fact that their phones are constantly scanning for WiFi hotspots and Bluetooth devices anyway. When they learn that the MAC address is a number on their device, they might confuse it with their phone number that others can use to call them or send text messages. When there are multiple visible beacons in a building that users come across along their way, it might lead to the impression that they are being traced and their route could be followed. The distinction between counting and tracking is crucial when communicating the technology's purpose to the user.
\begin{table*}
\caption{Privacy and perceived privacy for different methods}
\label{table_privacy}
\centering
\begin{tabular}{c || c || c}
\hline
\bfseries \textbf{Method} & \textbf{Privacy} & \bfseries\textbf{Perceived privacy} \\
\hline
Device-free (e.g. SCPL) & Yes & Yes\\
\hline
Sensor data & Yes & Yes \\
\hline
Real-time people counting from depth imagery & No & No \\
\hline
Counting people with multiple cameras & No & No \\
\hline
Device-based & Yes & No
\end{tabular}
\end{table*}

\section{Conclusion}
Complex wave propagation and decoherence make crowd counting using WiFi a challenging task that is yet to be solved without significant limitations. Scalability to a high number of people is difficult. Techniques refer to a diverse set of basic methods like the Collection Tree Protocol, Conditional Random Fields or Renewal theory. Assumptions made in research literature often do not translate to real-world scenarios. Device-based methods impose additional complications for privacy. Research is still experimental, emerging and ongoing.



\subsection{Authors and Affiliations}
\textbf{The class file is designed for, but not limited to, six authors.} A 
minimum of one author is required for all conference articles. Author names 
should be listed starting from left to right and then moving down to the 
next line. This is the author sequence that will be used in future citations 
and by indexing services. Names should not be listed in columns nor group by 
affiliation. Please keep your affiliations as succinct as possible (for 
example, do not differentiate among departments of the same organization).

\subsection{Identify the Headings}
Headings, or heads, are organizational devices that guide the reader through 
your paper. There are two types: component heads and text heads.

Component heads identify the different components of your paper and are not 
topically subordinate to each other. Examples include Acknowledgments and 
References and, for these, the correct style to use is ``Heading 5''. Use 
``figure caption'' for your Figure captions, and ``table head'' for your 
table title. Run-in heads, such as ``Abstract'', will require you to apply a 
style (in this case, italic) in addition to the style provided by the drop 
down menu to differentiate the head from the text.

Text heads organize the topics on a relational, hierarchical basis. For 
example, the paper title is the primary text head because all subsequent 
material relates and elaborates on this one topic. If there are two or more 
sub-topics, the next level head (uppercase Roman numerals) should be used 
and, conversely, if there are not at least two sub-topics, then no subheads 
should be introduced.

\subsection{Figures and Tables}
\paragraph{Positioning Figures and Tables} Place figures and tables at the top and 
bottom of columns. Avoid placing them in the middle of columns. Large 
figures and tables may span across both columns. Figure captions should be 
below the figures; table heads should appear above the tables. Insert 
figures and tables after they are cited in the text. Use the abbreviation 
``Fig.~\ref{fig}'', even at the beginning of a sentence.

\begin{table}[htbp]
\caption{Table Type Styles}
\begin{center}
\begin{tabular}{|c|c|c|c|}
\hline
\textbf{Table}&\multicolumn{3}{|c|}{\textbf{Table Column Head}} \\
\cline{2-4} 
\textbf{Head} & \textbf{\textit{Table column subhead}}& \textbf{\textit{Subhead}}& \textbf{\textit{Subhead}} \\
\hline
copy& More table copy$^{\mathrm{a}}$& &  \\
\hline
\multicolumn{4}{l}{$^{\mathrm{a}}$Sample of a Table footnote.}
\end{tabular}
\label{tab1}
\end{center}
\end{table}

\begin{figure}[htbp]
\centerline{\includegraphics[scale=0.2]{fig1.png}}
\caption{Example of a figure caption.}
\label{fig}
\end{figure}

Figure Labels: Use 8 point Times New Roman for Figure labels. Use words 
rather than symbols or abbreviations when writing Figure axis labels to 
avoid confusing the reader. As an example, write the quantity 
``Magnetization'', or ``Magnetization, M'', not just ``M''. If including 
units in the label, present them within parentheses. Do not label axes only 
with units. In the example, write ``Magnetization (A/m)'' or ``Magnetization 
\{A[m(1)]\}'', not just ``A/m''. Do not label axes with a ratio of 
quantities and units. For example, write ``Temperature (K)'', not 
``Temperature/K''.

\section*{Acknowledgment}

The preferred spelling of the word ``acknowledgment'' in America is without 
an ``e'' after the ``g''. Avoid the stilted expression ``one of us (R. B. 
G.) thanks $\ldots$''. Instead, try ``R. B. G. thanks$\ldots$''. Put sponsor 
acknowledgments in the unnumbered footnote on the first page.

\section*{References}

Please number citations consecutively within brackets. The 
sentence punctuation follows the bracket \cite{b2}. Refer simply to the reference 
number, as in \cite{b3}---do not use ``Ref. \cite{b3}'' or ``reference \cite{b3}'' except at 
the beginning of a sentence: ``Reference \cite{b3} was the first $\ldots$''

Number footnotes separately in superscripts. Place the actual footnote at 
the bottom of the column in which it was cited. Do not put footnotes in the 
abstract or reference list. Use letters for table footnotes.

Unless there are six authors or more give all authors' names; do not use 
``et al.''. Papers that have not been published, even if they have been 
submitted for publication, should be cited as ``unpublished'' \cite{b4}. Papers 
that have been accepted for publication should be cited as ``in press'' \cite{b5}. 
Capitalize only the first word in a paper title, except for proper nouns and 
element symbols.

For papers published in translation journals, please give the English 
citation first, followed by the original foreign-language citation \cite{b6}.

\begin{thebibliography}{00}
\bibitem{Agarwal} Y. Agarwal, B. Balaji, R. Gupta, J. Lyles, M. Wei, and T. Weng, 
"Occupancy-Driven Energy Management for Smart Building Automation," in Proceedings of the 2nd ACM Workshop on Embedded Sensing Systems for Energy-Efficiency in Building. ACM, 2010, pp. 1–6.
\bibitem{RetailStores} O. Perdikaki, S. Kesavan, and J. M. Swaminathan, "Effect of traffic on sales and conversion rates of retail stores," Manufacturing & Service Operations Management, vol. 14, no. 1, pp.145–162, 2012.
\bibitem{fcctable} Federal Communications Commission Office of Engineering and Technology Policy and Rules Division, "FCC ONLINE TABLE OF FREQUENCY ALLOCATIONS," 2018. [Online]. Available: https://transition.fcc.gov/oet/spectrum/table/fcctable.pdf. [Accessed: 08- Jun- 2018].
\bibitem{Yuan} Y. Yuan, C. Qiu, W. Xi, and J. Zhao, “Crowd density estimation using wireless sensor networks,” in Seventh International Conference on Mobile Ad-hoc and Sensor Networks. IEEE, 2011, pp. 138–145.


\bibitem{ctp} O. Gnawali, R. Fonseca, K. Jamieson, D. Moss, and P. Levis, "Collection Tree Protocol," In ACM Transactions on Sensor Networks (TOSN), Vol. 10, No. 3, 2013.
\bibitem{Xu} C. Xu, B. Firner, R. S. Moore, Y. Zhang, W. Trappe, R. Howard, F. Zhang, and N. An, "SCPL: indoor device-free multi-subject counting and localization using radio signal strength," in ACM/IEEE International Conference on Information Processing in Sensor Networks. IEEE, 2013, pp. 79–90.
\bibitem{crf} J. Lafferty, A. McCallum, and F. C.N. Pereira, "Conditional Random Fields: Probabilistic Models for Segmenting and Labeling Sequence Data," in Proceedings of the 18th International Conference on Machine Learning 2001 (ICML 2001), pages 282-289.

\bibitem{Xi} W. Xi, J. Zhao, X.-Y. Li, K. Zhao, S. Tang, X. Liu, and Z. Jiang, "Electronic frog eye: Counting crowd using WiFi," in IEEE INFOCOM 2014-IEEE Conference on Computer Communications, pp. 361–369.


\bibitem{Yoshida} T. Yoshida and Y. Taniguchi, "Estimating the number of people using existing WiFi access point in indoor environment," in Proceedings of the 6th European Conference of Computer Science, 2015, pp. 46–53.

\bibitem{DepatlaMostofi2015} S. Depatla, L. Buckland, and Y. Mostofi, "X-ray vision with only wifi power measurements using rytov wave models," IEEE Trans. on Vehicular Technology, vol. 64, no. 4, pp. 1376-1387, 2015.

\bibitem{DepatlaMostofi2018} S. Depatla and Y. Mostofi, "Crowd Counting Through Walls Using Wifi," to appear in IEEE International Conference on Pervasive Computing and Communications, 2018.
\bibitem{OpenWrt} [Online]. Available: https://openwrt.org/. [Accessed: 12- Jun- 2018].

\bibitem{crypto} [Online]. Available: https://arstechnica.com/information-technology/2013/06/gold-in-them-bits-inside-the-worlds-most-mysterious-bitcoin-mining-company/2/. [Accessed: 13- Jun- 2018].

\end{thebibliography}


\end{document}
